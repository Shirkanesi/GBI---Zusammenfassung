\documentclass{article}

% Sonderzeichen ermöglichen
\usepackage[ngerman]{babel}
\usepackage[utf8]{inputenc}
\usepackage[T1]{fontenc}
\usepackage{a4wide}
\usepackage{xcolor}

%Subitle
\usepackage{relsize}
\newcommand{\subtitlerelsize}{1} %relative size: integer value
\newcommand{\subtitlelinesep}{0.1em} %line separation: a LaTeX length

% Hyperlinks in das Dokument einfügen
\usepackage{hyperref}

% Mathe-Packages einbinden
\usepackage{amsmath}
\usepackage{amssymb}
\usepackage{amsthm}

\renewcommand\labelenumi{(\roman{enumi})}
\renewcommand\theenumi\labelenumi

%Farbdefinitionen
\definecolor{strongColor}{RGB}{189, 66, 0}
\definecolor{lightblue}{RGB}{0, 140, 232}
\definecolor{importantColor}{RGB}{163, 19, 0}
\definecolor{ForestGreen}{RGB}{34,139,34}

% Eigene Befehlsdefinitionen
\newcommand{\N}{\mathbb{N_+}} %Natürliche Zahlen (ohne 0)
\newcommand{\Nz}{{\mathbb{N}_0}} %Natürliche Zahlen (mit 0)
\newcommand{\Z}{\mathbb{Z}} % Ganze Zahlen
\newcommand{\Q}{\mathbb{Q}} % Rationale Zahlen
\newcommand{\R}{\mathbb{R}} % Reelle Zahlen
\newcommand{\vektor}[1]{\begin{pmatrix}#1\end{pmatrix}} %Vetoren

\newcommand{\kapitel}[2]{Kapitel #1 - #2}

\newcommand{\blue}[1]{\textcolor{blue}{#1}}
\newcommand{\red}[1]{\textcolor{red}{#1}}
\newcommand{\babyblue}[1]{\textcolor{lightblue}{#1}}

\newcommand{\strongColor}[1]{\textcolor{strongColor}{#1}}
\newcommand{\strong}[1]{\textbf{\strongColor{#1}}}

\newcommand{\important}[1]{\textcolor{importantColor}{#1}}
\newcommand{\verweis}[1]{\textcolor{ForestGreen}{#1}}

\newcommand{\example}[1]{\textit{Beispiel: }#1}
\newcommand{\word}[1]{\blue{\texttt{#1}}}
\newcommand{\interpretation}[1]{\babyblue{#1}}

\newcommand{\wsp}{\word{\tiny $\sqcup$}}

\newcommand{\Fbox}[1]{\fbox{\strut#1}}
\setlength{\fboxsep}{1pt}% Just for this example
\setlength{\parindent}{0pt}% Just for this example

\newcommand*{\mystrut}{\rule[-0.05\baselineskip]{0pt}{2\baselineskip}}
\newcommand*{\wordbox}[1]{\Fbox{#1}}


\newtheorem{satz}{Satz}
\newtheorem{aufgabe}{Aufgabe}
\newtheorem*{tipp*}{Tipp}
\setlength{\parindent}{0px}

\begin{document}
\title{Zusammenfassung GBI [im WS20/21]\\[\subtitlelinesep]%
    \smaller[\subtitlerelsize]{}Version 1}
\author{Julian Keck\\uwgmn}
%\date{10.12.2020}
\date{\today}

\maketitle

\tableofcontents

\newpage

\section{\kapitel{2}{Signale, Nachrichten, Informationen, Daten}}
\subsection{Signal}
\begin{itemize}
    \item Physikalische Vorgänge vermitteln einen \dq Eindruck\dq{} dessen, was mitgeteilt werden soll.
\end{itemize}

\subsection{Mitteilung}
\begin{itemize}
    \item Wird als Inschrift gespeichert
    \item Ein Gebilde, um etwas als Signal zu speichern
    \item Speichermethoden wie Höhle/Pinsel, Papier/Stift, Eisen/Magnet...
\end{itemize}

\subsection{Nachricht}
\begin{itemize}
    \item Das, was da steht, also unabhängig von Signal/Speichermethode ist.
    \item \example{\blue{10001}: Eins Null Null Null Eins}
    \item \red{10001} ist immernoch die gleiche Nachricht!
    \item Abstraktion der Mitteilung; \red{Un}abhängig von der Art der Speicherung/des Transports.
\end{itemize}

\subsection{Information}
\begin{itemize}
    \item Informationen erhält man durch \strong{Interpretation} einer Nachricht.
    \item Einer Nachricht wird eine Bedeutung zugeordnet
    \item Diese Interpretation erfolgt im \strong{Kopf} und nicht im Computer!
    \item Die Interpretation einer Nachricht kann unterschiedlich sein:\\\example
    \begin{itemize}
        \item \word{10001} interpretieren wir als \interpretation{zehntausendundeins}
        \item \word{10001} kann man auch als \interpretation{17} interpretieren (binär).
        \item oder auch als \interpretation{11} (Hexadezimaldarstellung der 17)
    \end{itemize}
    \item Der Rechner hat \dq keine Ahnung\dq{}, was er da tut, macht es aber trotzdem.
\end{itemize}

\subsection{Datum}
\begin{itemize}
    \item Singular von \strong{Daten}, nicht ein Tag im Jahr
    \item Das Bezugssystem der Interpretation ist relevant.
    \item \important{Auf eine Interpretation einigen (und diese festhalten)}
\end{itemize}

\newpage
\section{\kapitel{3}{Mengen, Alphabete, Abbildungen}}
\subsection{Mengen}
\begin{itemize}
    \item Eine Menge ist ein \dq Behälter\dq{} mit \dq Objekten\dq{}
    \item Diese Definition ist gefährlich \verweis{Russle's-Paradoxon}
    \item Eine Menge kann ein Objekt enthalten oder nicht (nicht beides und auch keines von beidem)
    \item \example{$\strongColor{A} = \{1,2,3\}$} ist eine Menge
    \item Auf Mengen kann man die bekannten Operationen wie Mengenschnitt ($\cap$), Mengenvereinigung ($\cup$) und Mengendifferenz ( $\setminus$ ) ausführen
    \item Teilmengenrelationen seien von jetzt an bekannt ($\subsetneq, \subseteq$); auf $\subset$ sollte verzichtet werden.
    \item Sei $A$ eine endliche Menge. Dann bezeichnet $|A|$ die \strong{Kardinalität} von $A$, also genau die Anzahl der Elemente in $A$
\end{itemize}

\subsection{Alphabet}
\begin{itemize}
    \item Ein Alphabet ist eine nichtleere Menge von Zeichen oder Symbolen
    \item Was ein Zeichen ist, sei nicht weiter spezifiziert, es handelt sich um einen elementaren Baustein von Inschriften
    \item \example{}
    \begin{itemize}
        \item Das deutsche Alphabet
        \item $A=\{\word{0,1}\}$ (\interpretation{Alphabet aller binären Worte})
        \item $A=\{\word{0,1,2,3,4,5,6,7,8,9,A,B,C,D,E,F}\}$ (\interpretation{Alphabet aller hexadezimalen Worte})
        \item ASCII
    \end{itemize}
    \item Manchmal sind Zeichen auch etwas abstrakter, \example{\word{\wordbox{int} \wordbox{counter} \wordbox{=} \wordbox{42} \wordbox{;}}}
\end{itemize}

\subsection{Paare}
\begin{itemize}
    \item Ein Paar $(a, b)$ hat die erste Komponente $a$ und die zweite $b$.
    \item Im Allgemeinen gilt: $(a, b) \neq (b, a)$
\end{itemize}

\subsection{Kartesisched Produkt}
\begin{itemize}
    \item Seien $A, B$ zwei Mengen
    \item Das \strong{kartesische Produkt $A \times B$} ist definiert als: $A \times B = \{(a,b)\mid a\in A \and b\in B\}$ 
    \item $A^2 := A \times A$
\end{itemize}

\newpage

\section{\kapitel{4}{Wörter (und Sprachen)}}
\subsection{Wörter}
\begin{itemize}
    \item \example
    \begin{itemize}
        \item \word{Apfelmus}
        \item \word{Milchreis} - Symbole dürfen also auch mehrfach vorkommen
    \end{itemize}
    \item Das Leerzeichen betrachten wir auch als Zeichen. Manchmal schreiben wir explizit \wsp
    \item \word{Hallo\wsp Welt} ist \important{eine} Folge von Zeichen, also \important{ein} Wort und nicht zwei.
\end{itemize}


\end{document}