\documentclass{article}

% Sonderzeichen ermöglichen
\usepackage[ngerman]{babel}
\usepackage[utf8]{inputenc}
\usepackage[T1]{fontenc}
\usepackage{a4wide}
\usepackage{xcolor}

%Subitle
\usepackage{relsize}
\newcommand{\subtitlerelsize}{1} %relative size: integer value
\newcommand{\subtitlelinesep}{0.1em} %line separation: a LaTeX length

% Hyperlinks in das Dokument einfügen
\usepackage{hyperref}

% Mathe-Packages einbinden
\usepackage{amsmath}
\usepackage{amssymb}
\usepackage{amsthm}

\renewcommand\labelenumi{(\roman{enumi})}
\renewcommand\theenumi\labelenumi

%Farbdefinitionen
\definecolor{strongColor}{RGB}{189, 66, 0}
\definecolor{lightblue}{RGB}{0, 140, 232}
\definecolor{importantColor}{RGB}{163, 19, 0}
\definecolor{ForestGreen}{RGB}{34,139,34}

% Eigene Befehlsdefinitionen
\newcommand{\N}{\mathbb{N_+}} %Natürliche Zahlen (ohne 0)
\newcommand{\Nz}{{\mathbb{N}_0}} %Natürliche Zahlen (mit 0)
\newcommand{\Z}{\mathbb{Z}} % Ganze Zahlen
\newcommand{\Q}{\mathbb{Q}} % Rationale Zahlen
\newcommand{\R}{\mathbb{R}} % Reelle Zahlen
\newcommand{\vektor}[1]{\begin{pmatrix}#1\end{pmatrix}} %Vetoren

\newcommand{\leer}{$\varepsilon$}

\newcommand{\kapitel}[2]{Kapitel #1 - \textsc{#2}}

\newcommand{\blue}[1]{\textcolor{blue}{#1}}
\newcommand{\red}[1]{\textcolor{red}{#1}}
\newcommand{\babyblue}[1]{\textcolor{lightblue}{#1}}

\newcommand{\strongColor}[1]{\textcolor{strongColor}{#1}}
\newcommand{\strong}[1]{\textbf{\strongColor{#1}}}

\newcommand{\important}[1]{\textcolor{importantColor}{#1}}
\newcommand{\verweis}[1]{\textcolor{ForestGreen}{#1}}

\newcommand{\example}[1]{\textit{Beispiel: }#1}
\newcommand{\word}[1]{\blue{\texttt{#1}}}
\newcommand{\interpretation}[1]{\babyblue{#1}}
\newcommand{\set}[1]{\{#1\}}
\newcommand{\anfuehrung}[1]{\flqq #1\frqq}

\newcommand{\w}{\textsf{w}}

\newcommand{\wsp}{\word{\tiny $\sqcup$}}

\newcommand{\Fbox}[1]{\fbox{\strut#1}}
\setlength{\fboxsep}{1pt}% Just for this example
\setlength{\parindent}{0pt}% Just for this example

\newcommand*{\mystrut}{\rule[-0.05\baselineskip]{0pt}{2\baselineskip}}
\newcommand*{\wordbox}[1]{\Fbox{#1}}


\newtheorem{satz}{Satz}
\newtheorem{aufgabe}{Aufgabe}
\newtheorem*{tipp*}{Tipp}
\setlength{\parindent}{0px}

\begin{document}
\title{Zusammenfassung GBI [im WS20/21]\\[\subtitlelinesep]%
    \smaller[\subtitlerelsize]{}Version 1}
\author{Julian Keck\\uwgmn}
%\date{10.12.2020}
\date{\today}

\maketitle

\tableofcontents

\newpage

\section{\kapitel{2}{Signale, Nachrichten, Informationen, Daten}}
\subsection{Signal}
\begin{itemize}
    \item Physikalische Vorgänge vermitteln einen \dq Eindruck\dq{} dessen, was mitgeteilt werden soll.
\end{itemize}

\subsection{Mitteilung}
\begin{itemize}
    \item Wird als Inschrift gespeichert
    \item Ein Gebilde, um etwas als Signal zu speichern
    \item Speichermethoden wie Höhle/Pinsel, Papier/Stift, Eisen/Magnet...
\end{itemize}

\subsection{Nachricht}
\begin{itemize}
    \item Das, was da steht, also unabhängig von Signal/Speichermethode ist.
    \item \example{\blue{10001}: Eins Null Null Null Eins}
    \item \red{10001} ist immernoch die gleiche Nachricht!
    \item Abstraktion der Mitteilung; \red{Un}abhängig von der Art der Speicherung/des Transports.
\end{itemize}

\subsection{Information}
\begin{itemize}
    \item Informationen erhält man durch \strong{Interpretation} einer Nachricht.
    \item Einer Nachricht wird eine Bedeutung zugeordnet
    \item Diese Interpretation erfolgt im \strong{Kopf} und nicht im Computer!
    \item Die Interpretation einer Nachricht kann unterschiedlich sein:\\\example
    \begin{itemize}
        \item \word{10001} interpretieren wir als \interpretation{zehntausendundeins}
        \item \word{10001} kann man auch als \interpretation{17} interpretieren (binär).
        \item oder auch als \interpretation{11} (Hexadezimaldarstellung der 17)
    \end{itemize}
    \item Der Rechner hat \dq keine Ahnung\dq{}, was er da tut, macht es aber trotzdem.
\end{itemize}

\subsection{Datum}
\begin{itemize}
    \item Singular von \strong{Daten}, nicht ein Tag im Jahr
    \item Das Bezugssystem der Interpretation ist relevant.
    \item \important{Auf eine Interpretation einigen (und diese festhalten)}
\end{itemize}

\newpage
\section{\kapitel{3}{Mengen, Alphabete, Abbildungen}}
\subsection{Mengen}
\begin{itemize}
    \item Eine Menge ist ein \dq Behälter\dq{} mit \dq Objekten\dq{}
    \item Diese Definition ist gefährlich \verweis{Russle's-Paradoxon}
    \item Eine Menge kann ein Objekt enthalten oder nicht (nicht beides und auch keines von beidem)
    \item \example{$\strongColor{A} = \{1,2,3\}$} ist eine Menge
    \item Auf Mengen kann man die bekannten Operationen wie Mengenschnitt ($\cap$), Mengenvereinigung ($\cup$) und Mengendifferenz ( $\setminus$ ) ausführen
    \item Teilmengenrelationen seien von jetzt an bekannt ($\subsetneq, \subseteq$); auf $\subset$ sollte verzichtet werden.
    \item Sei $A$ eine endliche Menge. Dann bezeichnet $|A|$ die \strong{Kardinalität} von $A$, also genau die Anzahl der Elemente in $A$
\end{itemize}

\subsection{Alphabet}
\begin{itemize}
    \item Ein Alphabet ist eine nichtleere Menge von Zeichen oder Symbolen
    \item Was ein Zeichen ist, sei nicht weiter spezifiziert, es handelt sich um einen elementaren Baustein von Inschriften
    \item \example{}
    \begin{itemize}
        \item Das deutsche Alphabet
        \item $A=\{\word{0,1}\}$ (\interpretation{Alphabet aller binären Worte})
        \item $A=\{\word{0,1,2,3,4,5,6,7,8,9,A,B,C,D,E,F}\}$ (\interpretation{Alphabet aller hexadezimalen Worte})
        \item ASCII
    \end{itemize}
    \item Manchmal sind Zeichen auch etwas abstrakter, \example{\word{\wordbox{int} \wordbox{counter} \wordbox{=} \wordbox{42} \wordbox{;}}}
\end{itemize}

\subsection{Paare}
\begin{itemize}
    \item Ein Paar $(a, b)$ hat die erste Komponente $a$ und die zweite $b$.
    \item Im Allgemeinen gilt: $(a, b) \neq (b, a)$
\end{itemize}

\subsection{Kartesisched Produkt}
\begin{itemize}
    \item Seien $A, B$ zwei Mengen
    \item Das \strong{kartesische Produkt $A \times B$} ist definiert als: $A \times B = \{(a,b)\mid a\in A \and b\in B\}$ 
    \item $A^2 := A \times A$
\end{itemize}

\newpage

\section{\kapitel{4}{Wörter (und Sprachen)}}
\subsection{Wörter}
\begin{itemize}
    \item \example
    \begin{itemize}
        \item \word{Apfelmus}
        \item \word{Milchreis} - Symbole dürfen also auch mehrfach vorkommen
    \end{itemize}
    \item Das Leerzeichen betrachten wir auch als Zeichen. Manchmal schreiben wir explizit \wsp
    \item \word{Hallo\wsp Welt} ist \important{eine} Folge von Zeichen, also \important{ein} Wort und nicht zwei.
    \item Formalere Definition eines Wortes:
    \begin{itemize}
        \item $\Z_n :=$ "die n kleinsten nichtnegativen ganzen Zahlen"\\
        $\Z_n := \set{i \in \N \mid 0\leq i < n}$\\
        $\Z_0 = \set{}, \Z_1 = \set{0}, \Z_4 = \set{0,1,2,3}$
        \item Ein Wort über dem Alphabet $A$ ist eine \important{surjektive Abbildung}:\\
        $\w: \Z_n \to B$  mit $B\subseteq A$ 
        \item \important{$|w| := n$} ist die Länge des Wortes
        \item Das Wort der Länge 0 wird als \important{leeres Wort} oder \important{\leer} bezeichnet.
    \end{itemize}
    \item Die Menge aller Wörter der Länge $n$ wird als \important{$A^n$} geschrieben.\\\example
    \begin{itemize}
        \item $A^0=\{$\leer$\}$
        \item $A^1 = A = \set{\word{a,b}}$
        \item $A^3 = \set{\word{aaa,aab,aba,abb,baa,bab,bba,bbb}}$
    \end{itemize}
    \item Die Menge aller Wörter über dem Alphabet $A$ wird als $A^*$ geschrieben\\
    oder $\displaystyle{A^* = \bigcup_{i\in \Nz}A^i}$, wenn man eine Notation ohne Pünktchen bevorzugt.
    \item Die Konkatenation bezeichnet die Aneinanderreihung mehrerer Worte, geschrieben wird diese durch \strong{$\cdot$}\\
    \example\\ $x = \word{Hallo}, y = \word{Welt}$\\
    $xy:=x\cdot y = \word{HalloWelt}$
    \item Die Konkatenation ist im Allgemeinen \red{nicht} kommutativ.\\ \example\\ $\word{SCHRANK} \cdot \word{SCHLÜSSEL} = \word{SCHRANKSCHLÜSSEL} \neq \word{SCHLÜSSELSCHRANK} = \word{SCHLÜSSEL} \cdot \word{SCHRANK}$
    \item Die Potenzen von Wörtern sind induktiv definiert durch:\\
    $\forall \:\w \in A^*:\quad$$\w^0=$ \leer\qquad$\w^{n+1}=\w^n\cdot \w$
    
    \subsection{Binäre Relationen}
    \begin{itemize}
        \item Eine \important{binäre Operation} auf einer Menge $M$ ist eine Abbildung $f: M\times M \to M$
        \item Oftmals wird eine Infixschreibweise benutzt, \example statt $+(3,8)=11$ schreibt man $3+8=11$
        \item Eine binäre Operation $\star$ ist \strong{kommutativ}, wenn gilt:
        \[ \forall x \in M \:\forall y \in M: \quad x\star y = y \star x \]
        \item Eine binäre Operation $\star$ ist \strong{assoziativ}, wenn gilt:
        \[ \forall x,y,z \in M: \quad (x\star y)\star z = x \star (y \star z) \]
    \end{itemize}
\end{itemize}

\newpage

\section{\kapitel{5}{Aussagenlogik}}
\subsection{Aussagen}
\begin{itemize}
    \item Aussagen sind \important{objektiv} wahr oder falsch \\\example
    \begin{itemize}
        \item \anfuehrung{Die Abbildung $f: \R \to \R: x \mapsto \sqrt{x}$ ist injektiv} \qquad \textbf{ wahr}
        \item \anfuehrung{Die Abbildung $f: \R \to \R: x \mapsto \sqrt{x}$ ist surjektiv} \qquad \textbf{falsch}
    \end{itemize}
    \item Es gibt auch scheinbar sinnvolle Aussagen, die in Wirklichkeit aber sinnlos sind:\\
    \anfuehrung{Ein Barbier ist ein Mann, der genau diejenigen Männer rasiert, die sich nicht selbst rasieren.} 
    \textit{Frage: Wer rasiert den Barbier?}
    \item Jede Aussage besitzt einen \important{eindeutigen Wahrheitswert} (nämlich \strong{wahr} oder \strong{falsch}).
\end{itemize}

\subsection{Aussagenlogische Konnektive}
Seien $P$ und $Q$ zwei Aussagen
\begin{itemize}
    \item \makebox[2.5cm][r]{\textbf{Negation:}} \makebox[2.75cm][r]{\anfuehrung{nicht $P$}:} \strong{$\lnot P$}
    \item \makebox[2.5cm][r]{\textbf{logisches Und:}} \makebox[2.75cm][r]{\anfuehrung{$P$ und $Q$}:} \strong{$P \land Q$}
    \item \makebox[2.5cm][r]{\textbf{logisches Oder:}} \makebox[2.75cm][r]{\anfuehrung{$P$ oder $Q$}:} \strong{$P \lor Q$}
    \item \makebox[2.5cm][r]{\textbf{Implikation:}} \makebox[2.75cm][r]{\anfuehrung{$P$ impliziert $Q$}:} \strong{$P \rightarrow Q$}\\\makebox[5.25cm][r]{\anfuehrung{Wenn P, dann Q}}
\end{itemize}
Der Wahrheitswert einer \important{zusammengesetzten Aussage} ist durch die Wahrheitswerte der \important{Teilaussagen} \strong{eindeutig} festgelegt. Es gibt keine Abhängigkeit vom konkreten Inhalt der Aussagen, auch nicht bei \anfuehrung{Wenn..., dann...}.

\example
\begin{itemize}
    \item $P:$ \anfuehrung{2014 wurden in Japan etwa 4,7 Mio. PKW neu zugelassen}
    \item $Q:$ \anfuehrung{1999 gab es in Deutschland etwa 11,2 Mio. Internet-Nutzer}
    \item \anfuehrung{Wenn $P$, dann $Q$} ist wahr, da die Teilaussagen wahr sind.
\end{itemize}
Definition des Alphabet der aussagenlogischen Formeln:\\\qquad $A_{AL} := \set{\word{(}, \word{)}, \word{$\lnot$}, \word{$\land$}, \word{$\lor$}, \word{$\rightarrow$}} \cup \set{$Aussagevariablen$}$

Aussagenlogische Formeln sind die Worte aus $A^*_{AL}$, die sinnvoll geklammert und formuliert sind, dies wird im Weiteren nicht betrachtet.

\subsection{Boolsche Funktionen}
\begin{itemize}
    \item Aussagenlogische Formeln sind erstmal nur Zeichenketten.
    \item Sogenannte \important{boolsche Funktionen}:\\
    $f: \mathbb{B}^k \to \mathbb{B}$, \qquad wobei $\mathbb{B}:= \set{\textbf{w}, \textbf{f}}$\\
    sind Funktionen, die Wahrheitswerte auf Wahrheitswerte abbilden. Für die vorher definierten aussagenlogischen Konnektive gibt es die entsprechenden boolschen Funktionen:\\
    $b_{\word{$\lnot$}}(x_1), b_{\word{$\land$}}(x_1, x_2), b_{\word{$\lor$}}(x_1, x_2), b_{\word{$\rightarrow$}}(x_1, x_2)$, wobei diese im Regelfall als Infix notiert werden.
\end{itemize}

\subsection{Bedeutung und Interpretation aussagenlogischer Formeln}
\begin{itemize}
    \item Sei $V$ die Menge von Aussagevariablen
    \item \important{Interpretation} $I: V \to \mathbb{B}$\\
    Dann ist $\mathbb{B}^V$ die Menge aller Interpretationen
    \item Für jede aussagenlogische Formel $F$ ist die \strong{Auswertung} \important{$val_I(F)$} folgendermaßen definiert:
    \begin{align*}
        val_I(\word{$\lnot$} G) &= b_{\word{$\lnot$}}(val_I(G)) \\
        val_I(G\word{$\land$}H) &= b_{\word{$\land$}}(val_I(G), val_I(H)) \\
        val_I(G\word{$\lor$}H) &= b_{\word{$\lor$}}(val_I(G), val_I(H)) \\
        val_I(G\word{$\rightarrow$}H) &= b_{\word{$\rightarrow$}}(val_I(G), val_I(H)) \\
    \end{align*}
    
    \item Zwei Formeln $G$ und $H$ heißen \important{äquivalent}, wenn für jede Interpretation $I$ gilt:\\
    $val_I(G) = val_I(H)$\\
    \example
    \begin{itemize}
        \item \word{$\lnot$P$\lor \lnot$Q} und \word{$\lnot$(P$\land$Q)}
        \item \word{$\lnot \lnot$P} und \word{P}
    \end{itemize}
    \item Randbemerkung:\\
    G=$\word{P}_{\word{0}} \word{$\land$} \word{P}_{\word{0}}$ und H=$\word{P}_{\word{2}} \word{$\land$} \word{P}_{\word{2}}$ sind nicht äquivalent, da $\word{P}_{\word{0}}$ und $\word{P}_{\word{2}}$ nicht in jeder Interpretation $I$ den selben Wert haben.
    
\end{itemize}



\end{document}